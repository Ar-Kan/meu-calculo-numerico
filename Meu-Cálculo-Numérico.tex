% Options for packages loaded elsewhere
\PassOptionsToPackage{unicode}{hyperref}
\PassOptionsToPackage{hyphens}{url}
\PassOptionsToPackage{dvipsnames,svgnames,x11names}{xcolor}
%
\documentclass[
  a4paper,
  DIV=11,
  numbers=noendperiod,
  oneside]{scrreprt}

\usepackage{amsmath,amssymb}
\usepackage{iftex}
\ifPDFTeX
  \usepackage[T1]{fontenc}
  \usepackage[utf8]{inputenc}
  \usepackage{textcomp} % provide euro and other symbols
\else % if luatex or xetex
  \usepackage{unicode-math}
  \defaultfontfeatures{Scale=MatchLowercase}
  \defaultfontfeatures[\rmfamily]{Ligatures=TeX,Scale=1}
\fi
\usepackage{lmodern}
\ifPDFTeX\else  
    % xetex/luatex font selection
\fi
% Use upquote if available, for straight quotes in verbatim environments
\IfFileExists{upquote.sty}{\usepackage{upquote}}{}
\IfFileExists{microtype.sty}{% use microtype if available
  \usepackage[]{microtype}
  \UseMicrotypeSet[protrusion]{basicmath} % disable protrusion for tt fonts
}{}
\makeatletter
\@ifundefined{KOMAClassName}{% if non-KOMA class
  \IfFileExists{parskip.sty}{%
    \usepackage{parskip}
  }{% else
    \setlength{\parindent}{0pt}
    \setlength{\parskip}{6pt plus 2pt minus 1pt}}
}{% if KOMA class
  \KOMAoptions{parskip=half}}
\makeatother
\usepackage{xcolor}
\usepackage[top=2cm,bottom=2cm,head=1cm,foot=1cm,left=2cm,marginparwidth=4cm,textwidth=12cm,marginparsep=1cm,bindingoffset=0.5cm]{geometry}
\setlength{\emergencystretch}{3em} % prevent overfull lines
\setcounter{secnumdepth}{5}
% Make \paragraph and \subparagraph free-standing
\makeatletter
\ifx\paragraph\undefined\else
  \let\oldparagraph\paragraph
  \renewcommand{\paragraph}{
    \@ifstar
      \xxxParagraphStar
      \xxxParagraphNoStar
  }
  \newcommand{\xxxParagraphStar}[1]{\oldparagraph*{#1}\mbox{}}
  \newcommand{\xxxParagraphNoStar}[1]{\oldparagraph{#1}\mbox{}}
\fi
\ifx\subparagraph\undefined\else
  \let\oldsubparagraph\subparagraph
  \renewcommand{\subparagraph}{
    \@ifstar
      \xxxSubParagraphStar
      \xxxSubParagraphNoStar
  }
  \newcommand{\xxxSubParagraphStar}[1]{\oldsubparagraph*{#1}\mbox{}}
  \newcommand{\xxxSubParagraphNoStar}[1]{\oldsubparagraph{#1}\mbox{}}
\fi
\makeatother

\usepackage{color}
\usepackage{fancyvrb}
\newcommand{\VerbBar}{|}
\newcommand{\VERB}{\Verb[commandchars=\\\{\}]}
\DefineVerbatimEnvironment{Highlighting}{Verbatim}{commandchars=\\\{\}}
% Add ',fontsize=\small' for more characters per line
\usepackage{framed}
\definecolor{shadecolor}{RGB}{241,243,245}
\newenvironment{Shaded}{\begin{snugshade}}{\end{snugshade}}
\newcommand{\AlertTok}[1]{\textcolor[rgb]{0.68,0.00,0.00}{#1}}
\newcommand{\AnnotationTok}[1]{\textcolor[rgb]{0.37,0.37,0.37}{#1}}
\newcommand{\AttributeTok}[1]{\textcolor[rgb]{0.40,0.45,0.13}{#1}}
\newcommand{\BaseNTok}[1]{\textcolor[rgb]{0.68,0.00,0.00}{#1}}
\newcommand{\BuiltInTok}[1]{\textcolor[rgb]{0.00,0.23,0.31}{#1}}
\newcommand{\CharTok}[1]{\textcolor[rgb]{0.13,0.47,0.30}{#1}}
\newcommand{\CommentTok}[1]{\textcolor[rgb]{0.37,0.37,0.37}{#1}}
\newcommand{\CommentVarTok}[1]{\textcolor[rgb]{0.37,0.37,0.37}{\textit{#1}}}
\newcommand{\ConstantTok}[1]{\textcolor[rgb]{0.56,0.35,0.01}{#1}}
\newcommand{\ControlFlowTok}[1]{\textcolor[rgb]{0.00,0.23,0.31}{\textbf{#1}}}
\newcommand{\DataTypeTok}[1]{\textcolor[rgb]{0.68,0.00,0.00}{#1}}
\newcommand{\DecValTok}[1]{\textcolor[rgb]{0.68,0.00,0.00}{#1}}
\newcommand{\DocumentationTok}[1]{\textcolor[rgb]{0.37,0.37,0.37}{\textit{#1}}}
\newcommand{\ErrorTok}[1]{\textcolor[rgb]{0.68,0.00,0.00}{#1}}
\newcommand{\ExtensionTok}[1]{\textcolor[rgb]{0.00,0.23,0.31}{#1}}
\newcommand{\FloatTok}[1]{\textcolor[rgb]{0.68,0.00,0.00}{#1}}
\newcommand{\FunctionTok}[1]{\textcolor[rgb]{0.28,0.35,0.67}{#1}}
\newcommand{\ImportTok}[1]{\textcolor[rgb]{0.00,0.46,0.62}{#1}}
\newcommand{\InformationTok}[1]{\textcolor[rgb]{0.37,0.37,0.37}{#1}}
\newcommand{\KeywordTok}[1]{\textcolor[rgb]{0.00,0.23,0.31}{\textbf{#1}}}
\newcommand{\NormalTok}[1]{\textcolor[rgb]{0.00,0.23,0.31}{#1}}
\newcommand{\OperatorTok}[1]{\textcolor[rgb]{0.37,0.37,0.37}{#1}}
\newcommand{\OtherTok}[1]{\textcolor[rgb]{0.00,0.23,0.31}{#1}}
\newcommand{\PreprocessorTok}[1]{\textcolor[rgb]{0.68,0.00,0.00}{#1}}
\newcommand{\RegionMarkerTok}[1]{\textcolor[rgb]{0.00,0.23,0.31}{#1}}
\newcommand{\SpecialCharTok}[1]{\textcolor[rgb]{0.37,0.37,0.37}{#1}}
\newcommand{\SpecialStringTok}[1]{\textcolor[rgb]{0.13,0.47,0.30}{#1}}
\newcommand{\StringTok}[1]{\textcolor[rgb]{0.13,0.47,0.30}{#1}}
\newcommand{\VariableTok}[1]{\textcolor[rgb]{0.07,0.07,0.07}{#1}}
\newcommand{\VerbatimStringTok}[1]{\textcolor[rgb]{0.13,0.47,0.30}{#1}}
\newcommand{\WarningTok}[1]{\textcolor[rgb]{0.37,0.37,0.37}{\textit{#1}}}

\providecommand{\tightlist}{%
  \setlength{\itemsep}{0pt}\setlength{\parskip}{0pt}}\usepackage{longtable,booktabs,array}
\usepackage{calc} % for calculating minipage widths
% Correct order of tables after \paragraph or \subparagraph
\usepackage{etoolbox}
\makeatletter
\patchcmd\longtable{\par}{\if@noskipsec\mbox{}\fi\par}{}{}
\makeatother
% Allow footnotes in longtable head/foot
\IfFileExists{footnotehyper.sty}{\usepackage{footnotehyper}}{\usepackage{footnote}}
\makesavenoteenv{longtable}
\usepackage{graphicx}
\makeatletter
\newsavebox\pandoc@box
\newcommand*\pandocbounded[1]{% scales image to fit in text height/width
  \sbox\pandoc@box{#1}%
  \Gscale@div\@tempa{\textheight}{\dimexpr\ht\pandoc@box+\dp\pandoc@box\relax}%
  \Gscale@div\@tempb{\linewidth}{\wd\pandoc@box}%
  \ifdim\@tempb\p@<\@tempa\p@\let\@tempa\@tempb\fi% select the smaller of both
  \ifdim\@tempa\p@<\p@\scalebox{\@tempa}{\usebox\pandoc@box}%
  \else\usebox{\pandoc@box}%
  \fi%
}
% Set default figure placement to htbp
\def\fps@figure{htbp}
\makeatother
% definitions for citeproc citations
\NewDocumentCommand\citeproctext{}{}
\NewDocumentCommand\citeproc{mm}{%
  \begingroup\def\citeproctext{#2}\cite{#1}\endgroup}
\makeatletter
 % allow citations to break across lines
 \let\@cite@ofmt\@firstofone
 % avoid brackets around text for \cite:
 \def\@biblabel#1{}
 \def\@cite#1#2{{#1\if@tempswa , #2\fi}}
\makeatother
\newlength{\cslhangindent}
\setlength{\cslhangindent}{1.5em}
\newlength{\csllabelwidth}
\setlength{\csllabelwidth}{3em}
\newenvironment{CSLReferences}[2] % #1 hanging-indent, #2 entry-spacing
 {\begin{list}{}{%
  \setlength{\itemindent}{0pt}
  \setlength{\leftmargin}{0pt}
  \setlength{\parsep}{0pt}
  % turn on hanging indent if param 1 is 1
  \ifodd #1
   \setlength{\leftmargin}{\cslhangindent}
   \setlength{\itemindent}{-1\cslhangindent}
  \fi
  % set entry spacing
  \setlength{\itemsep}{#2\baselineskip}}}
 {\end{list}}
\usepackage{calc}
\newcommand{\CSLBlock}[1]{\hfill\break\parbox[t]{\linewidth}{\strut\ignorespaces#1\strut}}
\newcommand{\CSLLeftMargin}[1]{\parbox[t]{\csllabelwidth}{\strut#1\strut}}
\newcommand{\CSLRightInline}[1]{\parbox[t]{\linewidth - \csllabelwidth}{\strut#1\strut}}
\newcommand{\CSLIndent}[1]{\hspace{\cslhangindent}#1}

\KOMAoption{captions}{tableheading}
\makeatletter
\@ifpackageloaded{tcolorbox}{}{\usepackage[skins,breakable]{tcolorbox}}
\@ifpackageloaded{fontawesome5}{}{\usepackage{fontawesome5}}
\definecolor{quarto-callout-color}{HTML}{909090}
\definecolor{quarto-callout-note-color}{HTML}{0758E5}
\definecolor{quarto-callout-important-color}{HTML}{CC1914}
\definecolor{quarto-callout-warning-color}{HTML}{EB9113}
\definecolor{quarto-callout-tip-color}{HTML}{00A047}
\definecolor{quarto-callout-caution-color}{HTML}{FC5300}
\definecolor{quarto-callout-color-frame}{HTML}{acacac}
\definecolor{quarto-callout-note-color-frame}{HTML}{4582ec}
\definecolor{quarto-callout-important-color-frame}{HTML}{d9534f}
\definecolor{quarto-callout-warning-color-frame}{HTML}{f0ad4e}
\definecolor{quarto-callout-tip-color-frame}{HTML}{02b875}
\definecolor{quarto-callout-caution-color-frame}{HTML}{fd7e14}
\makeatother
\makeatletter
\@ifpackageloaded{bookmark}{}{\usepackage{bookmark}}
\makeatother
\makeatletter
\@ifpackageloaded{caption}{}{\usepackage{caption}}
\AtBeginDocument{%
\ifdefined\contentsname
  \renewcommand*\contentsname{Índice}
\else
  \newcommand\contentsname{Índice}
\fi
\ifdefined\listfigurename
  \renewcommand*\listfigurename{Lista de Figuras}
\else
  \newcommand\listfigurename{Lista de Figuras}
\fi
\ifdefined\listtablename
  \renewcommand*\listtablename{Lista de Tabelas}
\else
  \newcommand\listtablename{Lista de Tabelas}
\fi
\ifdefined\figurename
  \renewcommand*\figurename{Figura}
\else
  \newcommand\figurename{Figura}
\fi
\ifdefined\tablename
  \renewcommand*\tablename{Tabela}
\else
  \newcommand\tablename{Tabela}
\fi
}
\@ifpackageloaded{float}{}{\usepackage{float}}
\floatstyle{ruled}
\@ifundefined{c@chapter}{\newfloat{codelisting}{h}{lop}}{\newfloat{codelisting}{h}{lop}[chapter]}
\floatname{codelisting}{Listagem}
\newcommand*\listoflistings{\listof{codelisting}{Lista de Listagens}}
\makeatother
\makeatletter
\makeatother
\makeatletter
\@ifpackageloaded{caption}{}{\usepackage{caption}}
\@ifpackageloaded{subcaption}{}{\usepackage{subcaption}}
\makeatother
\makeatletter
\@ifpackageloaded{sidenotes}{}{\usepackage{sidenotes}}
\@ifpackageloaded{marginnote}{}{\usepackage{marginnote}}
\makeatother
\newcounter{quartocallouttipno}
\newcommand{\quartocallouttip}[1]{\refstepcounter{quartocallouttipno}\label{#1}}

\usepackage{bookmark}

\IfFileExists{xurl.sty}{\usepackage{xurl}}{} % add URL line breaks if available
\urlstyle{same} % disable monospaced font for URLs
% Make links footnotes instead of hotlinks:
\DeclareRobustCommand{\href}[2]{#2\sidenote{\footnotesize \url{#1}}}
\hypersetup{
  pdftitle={Meu Cálculo Numérico},
  pdfauthor={Arquimedes Macedo; REAMAT},
  colorlinks=true,
  linkcolor={blue},
  filecolor={Maroon},
  citecolor={Blue},
  urlcolor={Blue},
  pdfcreator={LaTeX via pandoc}}


\title{Meu Cálculo Numérico}
\usepackage{etoolbox}
\makeatletter
\providecommand{\subtitle}[1]{% add subtitle to \maketitle
  \apptocmd{\@title}{\par {\large #1 \par}}{}{}
}
\makeatother
\subtitle{Um livro colaborativo}
\author{Arquimedes Macedo \and REAMAT}
\date{29/12/2024}

\begin{document}
\maketitle

\renewcommand*\contentsname{Índice}
{
\hypersetup{linkcolor=}
\setcounter{tocdepth}{2}
\tableofcontents
}

\bookmarksetup{startatroot}

\chapter*{Bem-Vindo}\label{bem-vindo}
\addcontentsline{toc}{chapter}{Bem-Vindo}

\markboth{Bem-Vindo}{Bem-Vindo}

\textbf{REAMAT} é um projeto de escrita colaborativa de recursos
educacionais abertos (REA) sobre tópicos de matemática e suas
aplicações.

Nosso objetivo é de fomentar o desenvolvimento de materiais didáticos
pela colaboração entre professores e alunos de universidades, institutos
de educação e demais interessados no estudo e na aplicação da matemática
nos mais diversos ramos da ciência e da tecnologia.

O sucesso do projeto depende da colaboração! Participe diretamente da
escrita dos recursos educacionais, dê sugestões ou avise-nos de erros e
imprecisões. Toda a colaboração é bem-vinda. Veja como participar aqui.

Nós preparamos uma série de ações para ajudá-lo a participar. Em
primeiro lugar, o acesso irrestrito aos materias pode se dar através
deste site.

\begin{tcolorbox}[enhanced jigsaw, breakable, left=2mm, opacityback=0, arc=.35mm, colframe=quarto-callout-note-color-frame, leftrule=.75mm, bottomrule=.15mm, rightrule=.15mm, colback=white, toprule=.15mm]
\begin{minipage}[t]{5.5mm}
\textcolor{quarto-callout-note-color}{\faInfo}
\end{minipage}%
\begin{minipage}[t]{\textwidth - 5.5mm}

Os códigos fontes e a documentação dos recursos estão disponíveis em
repositórios GitHub públicos.

\end{minipage}%
\end{tcolorbox}

\section*{Licença}\label{licenuxe7a}
\addcontentsline{toc}{section}{Licença}

\markright{Licença}

Nada disso estaria completo sem uma licença apropriada à colaboração.
Por isso, escolhemos disponibilizar os materiais sob licença Creative
Commons Atribuição-CompartilhaIgual 3.0 Não Adaptada (CC-BY-SA 3.0) . Ou
seja, você pode copiar, redistribuir, alterar e construir um novo
material para qualquer uso, inclusive comercial. Leia a licença para
mais informações.

\bookmarksetup{startatroot}

\chapter*{Prefácio}\label{prefuxe1cio}
\addcontentsline{toc}{chapter}{Prefácio}

\markboth{Prefácio}{Prefácio}

Este livro busca abordar os tópicos de um curso de introdução ao cálculo
numérico moderno oferecido a estudantes de matemática, física,
engenharias e outros. A ênfase é colocada na formulação de problemas,
implementação em computador da resolução e interpretação de resultados.
Pressupõe-se que o estudante domine conhecimentos e habilidades típicas
desenvolvidas em cursos de graduação de cálculo, álgebra linear e
equações diferenciais. Conhecimento prévio em linguagem de computadores
é fortemente recomendável, embora apenas técnicas elementares de
programação sejam realmente necessárias.

Os códigos computacionais dos métodos numéricos apresentados no livro
são implementados em uma abordagem didática. Isto é, temos o objetivo de
que a implementação em linguagem computacional auxilie o leitor no
aprendizado das técnicas numéricas apresentadas no livro. Implementações
computacionais eficientes de técnicas de cálculo numérico podem ser
obtidas na série de livros `'Numerical Recipes'\,', veja
(\citeproc{ref-numerical}{Press 2007}).

\section*{Linguagens Computacionais}\label{linguagens-computacionais}
\addcontentsline{toc}{section}{Linguagens Computacionais}

\markright{Linguagens Computacionais}

\subsection{Python}

A utilização da linguagem computacional
\href{https://www.python.org/}{Python}.

\subsection{Octave}

A utilização da linguagem computacional
\href{https://www.gnu.org/software/octave/}{GNU Octave}.

\subsection{Scilab}

A utilização do software livre \href{https://www.scilab.org/}{Scilab}.

\bookmarksetup{startatroot}

\chapter*{Introdução}\label{introduuxe7uxe3o}
\addcontentsline{toc}{chapter}{Introdução}

\markboth{Introdução}{Introdução}

Cálculo numérico é a disciplina que estuda as técnicas para a solução
aproximada de problemas matemáticos. Estas técnicas são de natureza
analítica e computacional. As principais preocupações normalmente
envolvem exatidão e desempenho.

Aliado ao aumento contínuo da capacidade de computação disponível, o
desenvolvimento de métodos numéricos tornou a simulação computacional de
problemas matemáticos uma prática usual nas mais diversas áreas
científicas e tecnológicas. As então chamadas simulações numéricas são
constituídas de um arranjo de vários esquemas numéricos dedicados a
resolver problemas específicos como, por exemplo: resolver equações
algébricas, resolver sistemas de equações lineares, interpolar e ajustar
pontos, calcular derivadas e integrais, resolver equações diferenciais
ordinárias, etc. Neste livro, abordamos o desenvolvimento, a
implementação, a utilização e os aspectos teóricos de métodos numéricos
para a resolução desses problemas.

Trabalharemos com problemas que abordam aspectos teóricos e de
utilização dos métodos estudados, bem como com problemas de interesse na
engenharia, na física e na matemática aplicada.

A necessidade de aplicar aproximações numéricas decorre do fato de que
esses problemas podem se mostrar intratáveis se dispomos apenas de meios
puramente analíticos, como aqueles estudados nos cursos de cálculo e
álgebra linear. Por exemplo, o teorema de Abel-Ruffini nos garante que
não existe uma fórmula algébrica, isto é, envolvendo apenas operações
aritméticas e radicais, para calcular as raízes de uma equação
polinomial de qualquer grau, mas apenas casos particulares:

\begin{itemize}
\tightlist
\item
  Simplesmente isolar a incógnita para encontrar a raiz de uma equação
  do primeiro grau;
\item
  Fórmula de Bhaskara para encontrar raízes de uma equação do segundo
  grau;
\item
  Fórmula de Cardano para encontrar raízes de uma equação do terceiro
  grau;
\item
  Existe expressão para equações de quarto grau;
\item
  Casos simplificados de equações de grau maior que 4 onde alguns
  coeficientes são nulos também podem ser resolvidos.
\end{itemize}

Equações não polinomiais podem ser ainda mais complicadas de resolver
exatamente, por exemplo:

\begin{equation}\phantomsection\label{eq-exemplo1}{
\cos(x) = x \quad \text{ou} \quad xe^x = 10
}\end{equation}

Para resolver o problema de valor inicial:

\begin{equation}\phantomsection\label{eq-exemplo2}{
\begin{aligned}
y' + xy &= x, \\
y(0) &= 2,
\end{aligned}
}\end{equation}

podemos usar o método de fator integrante e obtemos
{\(y = 1 + e^{-x^2/2}\).} No entanto, não parece possível encontrar uma
expressão fechada em termos de funções elementares para a solução exata
do problema de valor inicial dado por:

\begin{equation}\phantomsection\label{eq-exemplo3}{
\begin{aligned}
y' + xy &= e^{-y}, \\
y(0) &= 2.
\end{aligned}
}\end{equation}

Da mesma forma, resolvemos a integral:

\begin{equation}\phantomsection\label{eq-exemplo4}{
\int_1^2 x e^{-x^2} dx
}\end{equation}

pelo método da substituição e obtemos
{\(\frac{1}{2}(e^{-1} - e^{-4})\).} Porém a integral:

\begin{equation}\phantomsection\label{eq-exemplo5}{
\int_1^2 e^{-x^2} dx
}\end{equation}

não pode ser expressa analiticamente em termos de funções elementares,
como uma consequência do teorema de Liouville.

A maioria dos problemas envolvendo fenômenos reais produzem modelos
matemáticos cuja solução analítica é difícil (ou impossível) de obter,
mesmo quando provamos que a solução existe. Nesse curso propomos
calcular aproximações numéricas para esses problemas, que apesar de, em
geral, serem diferentes da solução exata, mostraremos que elas podem ser
bem próximas.

Para entender a construção de aproximações é necessário estudar como
funciona a aritmética implementada nos computadores e erros de
arredondamento. Como computadores, em geral, usam uma base binária para
representar números, começaremos falando em mudança de base.

\part{Capítulo 1: Representação de números e aritmética de máquina}

Neste capítulo, abordaremos formas de representar números reais em
computadores. Iniciamos com uma discussão sobre representação posicional
e mudança de base. Então, enfatizaremos a representação de números com
quantidade finita de dígitos, mais especificamente, as representações de
números inteiros, ponto fixo e ponto flutuante em computadores.

A representação de números e a aritmética em computadores levam aos
chamados erros de arredondamento e de truncamento. Ao final deste
capítulo, abordaremos os efeitos do erro de arredondamento na computação
científica.

\chapter{Sistema de numeração e mudança de
base}\label{sec-sistema-de-numeracao-e-mudanca-de-base}

Usualmente, utilizamos o sistema de numeração decimal, isto é, base 10,
para representar números. Esse é um sistema de numeração em que a
posição do algarismo indica a potência de {\(10\)} pela qual seu valor é
multiplicado.

\begin{tcolorbox}[enhanced jigsaw, breakable, left=2mm, leftrule=.75mm, title=\textcolor{quarto-callout-note-color}{\faInfo}\hspace{0.5em}{Exemplo}, opacitybacktitle=0.6, titlerule=0mm, opacityback=0, arc=.35mm, coltitle=black, colframe=quarto-callout-note-color-frame, bottomtitle=1mm, bottomrule=.15mm, rightrule=.15mm, toptitle=1mm, toprule=.15mm, colback=white, colbacktitle=quarto-callout-note-color!10!white]

O número {\(293\)} é decomposto como:

\begin{equation}\phantomsection\label{eq-exemplo-1-sistema-de-numeracao}{
\begin{split}
  293 &= 2\ \text{centenas} + 9\ \text{dezenas }+ 3\ \text{unidades}\\
  &= 2\cdot 10^2+9\cdot 10^1+3\cdot 10^0
\end{split}
}\end{equation}

293 é igual a 2 centenas mais 9 dezenas mais 3 unidades. Que também é
igual a 2 vezes 10 elevado à potência 2, mais 9 vezes 10 elevado à
potência 1, mais 3 vezes 10 elevado à potência 0

\end{tcolorbox}

O sistema de numeração posicional também pode ser usado com outras
bases. Vejamos a seguinte definição.

\begin{tcolorbox}[enhanced jigsaw, breakable, left=2mm, leftrule=.75mm, title=\textcolor{quarto-callout-tip-color}{\faLightbulb}\hspace{0.5em}{Sistema de numeração de base \(b\)}, opacitybacktitle=0.6, titlerule=0mm, opacityback=0, arc=.35mm, coltitle=black, colframe=quarto-callout-tip-color-frame, bottomtitle=1mm, bottomrule=.15mm, rightrule=.15mm, toptitle=1mm, toprule=.15mm, colback=white, colbacktitle=quarto-callout-tip-color!10!white]

Dado um número natural {\(b>1\)} e o conjunto de símbolos {
\(\pmb{\pm},  \pmb{0}, \pmb{1}, \pmb{2},\dotsc, \pmb{b-1}\) }
\footnotemark{}, a sequência de símbolos

\[
\left(d_nd_{n-1} \cdots d_1d_0,d_{-1}d_{-2} \cdots \right)_b
\]

representa o número positivo

\[
d_n\cdot b^n + d_{n-1}\cdot b^{n-1} + \cdots + d_0\cdot b^0 + d_{-1}\cdot b^{-1}+d_{-2}\cdot b^{-2} + \cdots
\]

Para representar números negativos usamos o símbolo {\(-\)} à esquerda
do numeral\footnotemark{}.

\end{tcolorbox}

\footnotetext{Para {\(b>10\),} veja
Tip~\ref{tip-obs_sistema_de_numeracao}.}

\footnotetext{O uso do símbolo {\(+\)} é opcional na representação de
números positivos.}

\begin{tcolorbox}[enhanced jigsaw, breakable, left=2mm, leftrule=.75mm, title=\textcolor{quarto-callout-note-color}{\faInfo}\hspace{0.5em}{Nota \ref*{tip-obs_sistema_de_numeracao}: Observação}, opacitybacktitle=0.6, titlerule=0mm, opacityback=0, arc=.35mm, coltitle=black, colframe=quarto-callout-note-color-frame, bottomtitle=1mm, bottomrule=.15mm, rightrule=.15mm, toptitle=1mm, toprule=.15mm, colback=white, colbacktitle=quarto-callout-note-color!10!white]

\quartocallouttip{tip-obs_sistema_de_numeracao} 

Para sistemas de numeração com base {\(b \geq 10\)} é usual utilizar as
seguintes notações:

\begin{itemize}
\tightlist
\item
  No sistema de numeração decimal (\(b=10\)), costumamos representar o
  número sem os parênteses e o subíndice, ou seja,
\end{itemize}

\[
  \pm d_nd_{n-1}\ldots d_1d_0,d_{-1}d_{-2}\ldots := \pm (d_nd_{n-1}\ldots d_1d_0,d_{-1}d_{-2}\ldots)_{10}
  \]

\begin{itemize}
\tightlist
\item
  Se \(b>10\), usamos as letras \(A, B, C, \cdots\) para denotar os
  algarismos: \(A=10\), \(B=11\), \(C=12\), \(D=13\), \(E=14\),
  \(F=15\).
\end{itemize}

\end{tcolorbox}

\subsection{Python}

\begin{Shaded}
\begin{Highlighting}[]
\OperatorTok{\textgreater{}\textgreater{}\textgreater{}} \DecValTok{1}\OperatorTok{*}\DecValTok{2}\OperatorTok{**}\DecValTok{3} \OperatorTok{+} \DecValTok{0}\OperatorTok{*}\DecValTok{2}\OperatorTok{**}\DecValTok{2} \OperatorTok{+} \DecValTok{0}\OperatorTok{*}\DecValTok{2}\OperatorTok{**}\DecValTok{1} \OperatorTok{+} \DecValTok{1}\OperatorTok{*}\DecValTok{2}\OperatorTok{**}\DecValTok{0} \OperatorTok{+} \DecValTok{1}\OperatorTok{*}\DecValTok{2}\OperatorTok{**{-}}\DecValTok{1} \OperatorTok{+} \DecValTok{0}\OperatorTok{*}\DecValTok{2}\OperatorTok{**{-}}\DecValTok{2} \OperatorTok{+} \DecValTok{1}\OperatorTok{*}\DecValTok{2}\OperatorTok{**{-}}\DecValTok{3}
\FloatTok{9.625}
\end{Highlighting}
\end{Shaded}

\subsection{Scilab}

\begin{Shaded}
\begin{Highlighting}[]
\NormalTok{{-}{-}\textgreater{} }\DecValTok{1}\NormalTok{*}\DecValTok{2}\NormalTok{\^{}}\DecValTok{3}\NormalTok{ + }\DecValTok{0}\NormalTok{*}\DecValTok{2}\NormalTok{\^{}}\DecValTok{2}\NormalTok{ + }\DecValTok{0}\NormalTok{*}\DecValTok{2}\NormalTok{\^{}}\DecValTok{1}\NormalTok{ + }\DecValTok{1}\NormalTok{*}\DecValTok{2}\NormalTok{\^{}}\DecValTok{0}\NormalTok{ + }\DecValTok{1}\NormalTok{*}\DecValTok{2}\NormalTok{\^{}{-}}\DecValTok{1}\NormalTok{ + }\DecValTok{0}\NormalTok{*}\DecValTok{2}\NormalTok{\^{}{-}}\DecValTok{2}\NormalTok{ + }\DecValTok{1}\NormalTok{*}\DecValTok{2}\NormalTok{\^{}{-}}\DecValTok{3}
\FunctionTok{ans}\NormalTok{ =  }\FloatTok{9.6250}
\end{Highlighting}
\end{Shaded}

\subsection{Octave}

\begin{Shaded}
\begin{Highlighting}[]
\OperatorTok{\textgreater{}\textgreater{}} \FloatTok{1}\OperatorTok{*}\FloatTok{2}\OperatorTok{\^{}}\FloatTok{3} \OperatorTok{+} \FloatTok{0}\OperatorTok{*}\FloatTok{2}\OperatorTok{\^{}}\FloatTok{2} \OperatorTok{+} \FloatTok{0}\OperatorTok{*}\FloatTok{2}\OperatorTok{\^{}}\FloatTok{1} \OperatorTok{+} \FloatTok{1}\OperatorTok{*}\FloatTok{2}\OperatorTok{\^{}}\FloatTok{0} \OperatorTok{+} \FloatTok{1}\OperatorTok{*}\FloatTok{2}\OperatorTok{\^{}{-}}\FloatTok{1} \OperatorTok{+} \FloatTok{0}\OperatorTok{*}\FloatTok{2}\OperatorTok{\^{}{-}}\FloatTok{2} \OperatorTok{+} \FloatTok{1}\OperatorTok{*}\FloatTok{2}\OperatorTok{\^{}{-}}\FloatTok{3}
\FunctionTok{ans} \OperatorTok{=}  \FloatTok{9.6250}
\end{Highlighting}
\end{Shaded}

\section{Exercícios}\label{exercuxedcios}

\subsection{Exercício}\label{exercuxedcio}

Escreva cada número dado para a base {\(b\).}

\begin{enumerate}
\def\labelenumi{\alph{enumi})}
\tightlist
\item
  \((45,1)_8\) para a base \(b=2\)
\item
  \((21,2)_8\) para a base \(b=16\)
\item
  \((1001,101)_2\) para a base \(b=8\)
\item
  \((1001,101)_2\) para a base \(b=16\)
\end{enumerate}

\begin{tcolorbox}[enhanced jigsaw, breakable, left=2mm, leftrule=.75mm, title=\textcolor{quarto-callout-warning-color}{\faExclamationTriangle}\hspace{0.5em}{Resposta}, opacitybacktitle=0.6, titlerule=0mm, opacityback=0, arc=.35mm, coltitle=black, colframe=quarto-callout-warning-color-frame, bottomtitle=1mm, bottomrule=.15mm, rightrule=.15mm, toptitle=1mm, toprule=.15mm, colback=white, colbacktitle=quarto-callout-warning-color!10!white]

\begin{enumerate}
\def\labelenumi{\alph{enumi})}
\tightlist
\item
  \textasciitilde{}\((100101,001)_2\)
\item
  \textasciitilde{}\((11,4)_{16}\)
\item
  \textasciitilde{}\((11,5)_8\)
\item
  \textasciitilde{}\((9,A)_{16}\)
\end{enumerate}

\end{tcolorbox}

\subsection{Exercício}\label{exercuxedcio-1}

Quantos algarismos são necessários para representar o número
{\(937163832173947\)} em base binária? E em base 7?

\begin{tcolorbox}[enhanced jigsaw, breakable, left=2mm, leftrule=.75mm, title=\textcolor{quarto-callout-tip-color}{\faLightbulb}\hspace{0.5em}{Dica}, opacitybacktitle=0.6, titlerule=0mm, opacityback=0, arc=.35mm, coltitle=black, colframe=quarto-callout-tip-color-frame, bottomtitle=1mm, bottomrule=.15mm, rightrule=.15mm, toptitle=1mm, toprule=.15mm, colback=white, colbacktitle=quarto-callout-tip-color!10!white]

Qual é o menor e o maior inteiro que pode ser escrito em dada base com
{\(N\)} algarismos?

\end{tcolorbox}

\begin{tcolorbox}[enhanced jigsaw, breakable, left=2mm, leftrule=.75mm, title=\textcolor{quarto-callout-warning-color}{\faExclamationTriangle}\hspace{0.5em}{Resposta}, opacitybacktitle=0.6, titlerule=0mm, opacityback=0, arc=.35mm, coltitle=black, colframe=quarto-callout-warning-color-frame, bottomtitle=1mm, bottomrule=.15mm, rightrule=.15mm, toptitle=1mm, toprule=.15mm, colback=white, colbacktitle=quarto-callout-warning-color!10!white]

\(50\); {\(18\).}

\end{tcolorbox}

\cleardoublepage
\phantomsection
\addcontentsline{toc}{part}{Apêndices}
\appendix

\chapter{Rápida introdução ao Python}\label{sec-appendix-starting}

Neste apêndice, discutiremos os principais aspectos da linguagem
computacional Python que são essenciais para uma boa leitura desta
versão do livro. O material aqui apresentado, é uma adaptação livre do
Apêndice A de (\citeproc{ref-CALSCI}{Todos os Colaboradores 2016}).

\section{Sobre a linguagem Python}\label{sobre-a-linguagem-python}

\href{https://www.python.org/}{Python} é uma linguagem de programação de
alto nível, interpretada e multi-paradigma. Lançada por
\href{https://gvanrossum.github.io//}{Guido van Rossum}\sidenote{\footnotesize Guido
  van Rossum, nascido em 1956, programador de computadores dos Países
  Baixos.} em 1991 é, atualmente, mantida de forma colaborativa e
aberta.

Para mais informações, consulte:

\begin{itemize}
\tightlist
\item
  Página oficial da linguagem Python: https://www.python.org/
\item
  Comunidade Python Brasil: http://wiki.python.org.br/
\end{itemize}

Para iniciantes, recomendamos o curso EAD gratuito no site
\href{https://www.codecademy.com/}{Codecademy}:

\url{https://www.codecademy.com/learn/python}

\subsection{Instalação e
execução}\label{instalauxe7uxe3o-e-execuuxe7uxe3o}

Para executar um código Python é necessário ter instalado um
interpretador para a linguagem. No \href{https://www.python.org/}{site
oficial do Python} estão disponíveis para \emph{download} os
interpretadores para vários sistemas operacionais, como Linux, Mac OS e
Windows. Muitas distribuições de Linux (Linux Mint, Ubuntu, etc.) têm o
Python no seu sistema de pacotes (incluindo documentação em várias
línguas).

Ao longo do texto, assumiremos que o leitor esteja usando um computar
rodando Linux. Para outros sistemas, pode ser necessário fazer algumas
adaptações.

\subsection{Usando Python}\label{usando-python}

O uso do Python pode ser feito de três formas básicas:

\begin{itemize}
\tightlist
\item
  usando um \texttt{console\ Python} de modo iterativo;
\item
  executando um código codigo.py no console Python;
\item
  executando um código Python \texttt{codigo.py} diretamente em
  terminal;
\end{itemize}

\subsection{Exercício}\label{exercuxedcio-2}

Considere o seguinte pseudocódigo:

\begin{verbatim}
s = "Olá, mundo!". (Sem imprimir na tela o resultado.)
saída(s). (Imprime na tela.)
\end{verbatim}

Implemente este pseudocódigo em Python:

\begin{enumerate}
\def\labelenumi{\alph{enumi})}
\tightlist
\item
  usando diretamente um console;
\item
  digitando seu código em um arquivo separado e executando-o no console
  Python com a função \texttt{execfile}.
\item
  digitando seu código em um arquivo separado e executando-o em terminal
  com o comando Python.
\end{enumerate}

\begin{tcolorbox}[enhanced jigsaw, breakable, left=2mm, leftrule=.75mm, title=\textcolor{quarto-callout-warning-color}{\faExclamationTriangle}\hspace{0.5em}{Resposta}, opacitybacktitle=0.6, titlerule=0mm, opacityback=0, arc=.35mm, coltitle=black, colframe=quarto-callout-warning-color-frame, bottomtitle=1mm, bottomrule=.15mm, rightrule=.15mm, toptitle=1mm, toprule=.15mm, colback=white, colbacktitle=quarto-callout-warning-color!10!white]

Seguem as soluções de cada item:

\begin{enumerate}
\def\labelenumi{\alph{enumi})}
\item
  No console temos:

\begin{Shaded}
\begin{Highlighting}[]
\OperatorTok{\textgreater{}\textgreater{}\textgreater{}}\NormalTok{ s }\OperatorTok{=} \StringTok{"Olá, mundo!"}
\OperatorTok{\textgreater{}\textgreater{}\textgreater{}} \BuiltInTok{print}\NormalTok{(s)}
\NormalTok{Olá, mundo}\OperatorTok{!}
\end{Highlighting}
\end{Shaded}

  Para sair do console, digite:

\begin{Shaded}
\begin{Highlighting}[]
\OperatorTok{\textgreater{}\textgreater{}\textgreater{}}\NormalTok{ quit()}
\end{Highlighting}
\end{Shaded}
\item
  Abra o editor de texto de sua preferência e digite o código:

\begin{Shaded}
\begin{Highlighting}[]
\CommentTok{\#!/usr/bin/env python}
\CommentTok{\# {-}*{-} coding: utf{-}8 {-}*{-}}

\NormalTok{s }\OperatorTok{=} \StringTok{\textquotesingle{}Olá\textquotesingle{}}
\BuiltInTok{print}\NormalTok{(s)}
\end{Highlighting}
\end{Shaded}

  Salve o arquivo como, por exemplo, \texttt{ola.py}. No console Python,
  digite:

\begin{Shaded}
\begin{Highlighting}[]
\OperatorTok{\textgreater{}\textgreater{}\textgreater{}} \BuiltInTok{execfile}\NormalTok{(}\StringTok{"ola.py"}\NormalTok{)}
\end{Highlighting}
\end{Shaded}
\item
  Abra o editor de texto de sua preferência e digite o código:

\begin{Shaded}
\begin{Highlighting}[]
\CommentTok{\#!/usr/bin/env python}
\CommentTok{\# {-}*{-} coding: utf{-}8 {-}*{-}}

\NormalTok{s }\OperatorTok{=} \StringTok{\textquotesingle{}Olá\textquotesingle{}}
\BuiltInTok{print}\NormalTok{(s)}
\end{Highlighting}
\end{Shaded}

  Salve o arquivo como, por exemplo, \texttt{ola.py}. No terminal,
  digite:

\begin{Shaded}
\begin{Highlighting}[]
\ExtensionTok{$}\NormalTok{ python ola.py}
\end{Highlighting}
\end{Shaded}
\end{enumerate}

\end{tcolorbox}

\phantomsection\label{refs}
\begin{CSLReferences}{1}{0}
\bibitem[\citeproctext]{ref-numerical}
Press, W. H. 2007. \emph{Numerical Recipes 3rd Edition: The Art of
Scientific Computing}. Cambridge University Press.
\url{https://books.google.com.br/books?id=1aAOdzK3FegC}.

\bibitem[\citeproctext]{ref-CALSCI}
Todos os Colaboradores. 2016. {``Cálculo Numérico - Um Livro
Colaborativo - Versão Com Scilab.''}

\end{CSLReferences}




\end{document}
